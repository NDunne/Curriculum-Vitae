\documentclass[11pt]{article}
\usepackage{xcolor}
\usepackage{tabularx}
\usepackage{enumitem}
\usepackage{titlesec}
\usepackage{nopageno}
\usepackage{tikz}

\usepackage[margin=0.5in]{geometry}

\setlist[itemize]{itemsep=2pt, topsep=3pt}
\newcommand{\dash}{\item[-]}
\newcommand{\linesep}{\noindent\makebox[\linewidth]{\rule{\linewidth}{0.2pt}}}

\newcommand{\skillbox}[1]{\def\boxwidth {#1 * 7} \begin{tikzpicture}\draw[fill=darkgray] rectangle (\boxwidth pt, 5pt);\end{tikzpicture}\begin{tikzpicture}\draw[draw=darkgray, fill=white] rectangle (70 - \boxwidth pt, 5pt);\end{tikzpicture}}

 \renewcommand{\familydefault}{\sfdefault}

\pagestyle{plain}

\hyphenpenalty=10000
\exhyphenpenalty=10000 

\titleformat{\section}{\Huge\bfseries}{\thesection}{1em}{}
\titlespacing*{\section}{0pt}{2pt}{3pt}

\titleformat*{\subsection}{\Large\bfseries}
\titlespacing*{\subsection}{0pt}{2pt}{3pt}

%Curriculum Vitae

\begin{document}
\hspace*{\fill} 24 Ottways Lane, Ashtead, Surrey, KT21 2NZ \\
\hspace*{\fill} N.Dunne@warwick.ac.uk | 07473451007
\section*{Nathan Dunne}

I am a graduate of Computer Science from The University of Warwick, finishing my final year in 2020. I took an intercalated year to work as a Software Development Intern at Sophos, and as an Embedded Software Engineer at Arm. I consider myself a dedicated and hardworking proffesional, with a positive attitude and a desire to improve that helps me to quickly develop new skills. I hope to find somewhere that will continue to challenge me, and allow me to use and expand on the knowledge and skills I have gained so far.

\linesep
\subsection*{Education}

%% OLD FORMAT %%
% \textbf{Second Year} \hfill 1st (72\%) \hfill June 2018 \\
% \textcolor{darkgray}{University of Warwick} \\
% \hspace*{4ex}Database Systems (81\%) Software Engineering (79\%), Adv. Computer Architecture (78\%)
% \\\\
% \textbf{First Year} \hfill \hspace{1em} 1st (72\%) \hfill June 2017 \\    
% \textcolor{darkgray}{University of Warwick} \\
% \hspace*{4ex}Discrete Maths (85\%), Design of Information Structures (76\%), Intro to Java (72\%)
% \\\\
% \textbf{Advanced Level} \hfill August 2016 \\
% \textcolor{darkgray}{The City of London Freemen’s School (CLFS)} \\                    
% \hspace*{4ex}Maths (A*), Computing (A), Physics (A), Electronics (A) [Advanced Subsidiary]
% \\\\
% \textbf{General Certificate of Secondary Education} \hfill August 2014 \\
% \textcolor{darkgray}{The City of London Freemen’s School (CLFS)} \\  
% \hspace*{4ex}11 GCSEs A*-B, 7 at A* including Mathematics, Additional Mathematics and English Language
% \\

\newcommand\result[7]{\node[commit] (#1) {}; 
  \node[llabel] at (#1) {
  #3\\
  \textcolor{darkgray}{#6}\\
  \hspace*{4ex}#7
  };
  \node[rlabel, text width=\textwidth-#2em] at (#1) {#5};
}

\newcommand\subresult[6]{\node[commit] (#1) {}; 
  \node[llabel] at (#1) {
  #3\\
  \hspace*{4ex}#6
  };
  \node[rlabel, text width=\textwidth-#2em] at (#1) {#5};
}
\newcommand\ghost[1]{\coordinate (#1);}
\newcommand\connect[2]{\path (#1) to[out=90,in=-90] (#2);}

\begin{tikzpicture}

\tikzstyle{commit}=[draw,circle,fill=white,inner sep=0pt,minimum size=10pt]
\tikzstyle{llabel}=[right,outer sep=1em,align=left]
\tikzstyle{rlabel}=[right,outer sep=0pt,align=right, at start]
\tikzstyle{every path}=[draw]

\matrix [column sep={1em,between origins},row sep=7pt]
{
  \ghost{post} & \\
  \result{grad}{2}{Computer Science with Intercalated Year Graduate: First Class}{1st (72\%)}{July 2020}{University of Warwick}{Dissertation: Just-In-Time Compilation for an Unstructed Mesh DSL (77\%)} & \\
  \ghost{uni3} & \subresult{3rdYr}{2}{Third Year}{1st (73\%)}{2020}{Compiler Design (81\%), Fault-Tolerant Systems (77\%), Computer Graphics (75\%)}\\
  \ghost{uni2.5} & \subresult{int}{2}{Intercalated Year}{}{2019}{Sophos Ltd (11 months) \& Arm Ltd (3 months}{}\\
  \ghost{uni2} & \subresult{2ndYr}{2}{Second Year}{1st (72\%)}{2018}{Database Systems (81\%) Software Engineering (79\%), Adv. Computer Architecture (78\%)}\\
  \ghost{uni1} & \subresult{1stYr}{2}{First Year}{1st (72\%)}{2017}{Discrete Maths (85\%), Design of Information Structures (76\%), Intro to Java (72\%)}\\
  \result{AL}{1}{Advanced Level}{}{August 2016}{The City of London Freemen’s School (CLFS)}{Maths (A*), Computing (A), Physics (A), Electronics (A) [Advanced Subsidiary]} & \\
  \result{GCSE}{1}{General Certificate of Secondary Education}{}{August 2014}{The City of London Freemen’s School (CLFS)}{11 GCSEs A*-B, 7 at A* including Mathematics, Additional Mathematics and English Language} & \\
  \ghost{pre} & \\
};

\connect{grad}{post}
\connect{3rdYr}{grad}
\connect{uni3}{grad}
\connect{uni2}{uni2.5};
\connect{uni2.5}{uni3};
\connect{uni1}{uni2};
\connect{int}{3rdYr};
\connect{2ndYr}{int};
\connect{1stYr}{2ndYr};
\connect{AL}{uni1};
\connect{AL}{1stYr};
\connect{GCSE}{AL}
\connect{pre}{GCSE}

\end{tikzpicture}

 \linesep
 \subsection*{Employment}
\textbf{Summer Intern Embedded Software Engineer} \hfill June 2019 - August 2019 \\
\textcolor{darkgray}{ARM Ltd.}
  \begin{itemize}
   \dash Assigned to the ‘Clamshell’ team to work on optimisation of the Chromium web browser.
   \dash Took over an investigation from a graduate into a Google framework that could allow greater performance for the chromium binary based on its branching patterns.
   \dash This required me to become familiar with building the Linux Kernel, gain a deeper understanding of benchmarking the performance of software, and use an ARM hardware development board for the majority of my work. I also expanded my debugging knowledge with gdb/gdbserver, carrying over what I had learnt at Sophos and expanding on it.
  \end{itemize}
\vspace{10pt}
\textbf{Placement Year Software Engineer} \hfill July 2018 - June 2019 \\
\textcolor{darkgray}{Sophos Plc.}
  \begin{itemize}
   \dash Assigned to the ‘Fasttrack’ team to work as Development for Global Escalations Support.
   \dash Investigated customer and internal product issues when code analysis or a higher level of technical understanding was needed, requiring me to learn a number of debugging tools, as well as quickly become familiar with new code bases in .Net, C++ and C for different components.
   \dash Root causes where identified through live or static debugging, and analysis of logs generated by the product itself as well as troubleshooting tools e.g. Sysinternals, WireShark. In cases where a genuine bug existed I was sometimes able to implement fixes for the team that owned the product.
  \end{itemize}

  \linesep

  \subsection*{Technical Skills}
   \renewcommand\tabularxcolumn[1]{b{#1}}% for vertical centering text in X column

   \noindent\begin{tabularx}{\linewidth}{X X X X X X}
\textbf{Programming} & \textbf{Scripting} & \textbf{Source Control} & \textbf{Operating Systems} & \textbf{Productivity} & \textbf{Debugging} \\[10pt]
C++/C        & Python       & Git          & Windows       & Visual Studio    & gdb\\[1pt]
\skillbox{7} & \skillbox{8} & \skillbox{6} & \skillbox{8}  & \skillbox{6}     & \skillbox{6}\\[6pt]
Java         & Bash         & Perforce     & Linux/UNIX    & Microsoft Office & WinDBG \\
\skillbox{6} & \skillbox{8} & \skillbox{3} & \skillbox{7}  & \skillbox{8}     & \skillbox{5}\\[6pt]
JavaScript   & PowerShell   &              &               & LibreOffice      & SysInternals \\
\skillbox{4} & \skillbox{7} &              &               & \skillbox{8}     & \skillbox{6}\\[6pt]
Ada          & Lua          &              &               & LateX/BibTeX     & WireShark \\
\skillbox{3} & \skillbox{6} &              &               & \skillbox{6}     & \skillbox{4}\\[6pt]
   \end{tabularx}

%% OLD FORMAT %%
%    \vspace{10pt}
%    \noindent
%    \begin{tabularx}{\linewidth}{X X X}

% Windows & Visual Studio & gdb \\
% \skillbox{6} & \skillbox{8} & \skillbox{6} \\\\
% Linux/UNIX & Microsoft Office & WinDBG \\
% \skillbox{6} & \skillbox{8} & \skillbox{6} \\\\
% & LibreOffice & Windows SysInternals \\
% & \skillbox{8} & \skillbox{6} \\\\
% & LateX/BibTeX & WireShark \\
% & \skillbox{8} & \skillbox{6} \\\\
%    \end{tabularx}
  \linesep

  \subsection*{Volunteering}
\textbf{Project Leader} \hfill October 2016 - June 2018 \\
\textcolor{darkgray}{Warwick Technology Volunteers}
   \begin{itemize}
    \dash Planned, coordinated and lead sessions for school age students aimed at inspiring them to utilise technology creatively rather than simply consuming it. Our sessions involved either Scratch or Arduino as a platform, and usually consisted of assisting the students through a set of resources we’ve created.
	\dash Also advertised the project to fellow uni students to gain more volunteers and run more sessions, and provided training for the volunteers, to ensure a sufficient level of technical knowledge for them to lead school sessions themselves.
	\dash This greatly improved my confidence, communication, presentation giving, and teamwork \& leadership skills.
   \end{itemize}

  \linesep

  \subsection*{Interests}
  \begin{itemize}
\dash \textbf{Rock Climbing} \\ Indoor Rock Climbing and Bouldering 
\dash \textbf{Charity Fundraising} (Warwick RAG)
   \begin{itemize}[topsep=3pt, itemsep=2pt, partopsep=0pt, parsep=0pt]
    \item[] 3 Peaks challenge: \hfill $>$\textsterling 600 for \textit{The Meningitis Research Foundation}
    \item[] Mount Kilimanjaro: \hfill$>$\textsterling 3400 for \textit{The Meningitis Research Foundation}
    \item[] Skydive: \hfill $>$\textsterling 400 for \textit{Alziemers Research UK}
   \end{itemize}
\dash \textbf{Powerlifting} \\ \indent Warwick Barbell Member 2018 - present 
\dash \textbf{Technical Theatre} \\ \indent Lighting design for productions of \textit{Speed Death of the Radiant
Child}, \textit{Cabaret}; \\ Executive Committee member for Warwick Tech Crew 2017-2018.
  \end{itemize}

  \linesep

\titlespacing*{\subsection}{5pt}{2pt}{3pt}
\renewcommand{\arraystretch}{0.8}

   \vfill

   \subsection*{References}
     \begin{tabularx}{\linewidth}{ l X r }
Dr. Matthew Leeke && Dr. Arshad Jhumka \\
Department of Computer Science && Department of Computer Science \\
University of Warwick && University of Warwick \\
Tel: +44 (0) 2476523366 && Tel: +44 (0) 2476573780 \\
E-mail: matthew.leeke@warwick.ac.uk && E-mail: h.a.jhumka@warwick.ac.uk 
    \end{tabularx}
\end{document}
